

 % !TEX encoding = UTF-8 Unicode

\documentclass[a4paper]{report}

\usepackage[T2A]{fontenc} % enable Cyrillic fonts
\usepackage[utf8x,utf8]{inputenc} % make weird characters work
\usepackage[serbian]{babel}
%\usepackage[english,serbianc]{babel}
\usepackage{amssymb}

\usepackage{color}
\usepackage{url}
\usepackage[unicode]{hyperref}
\hypersetup{colorlinks,citecolor=green,filecolor=green,linkcolor=blue,urlcolor=blue}

\newcommand{\odgovor}[1]{\textcolor{blue}{#1}}

\begin{document}

\title{Projekat GraalVM\\ \small{Bojan Bardžić, Milica Gnjatović, Pavle Savić, Andrija Urošević}}

\maketitle

\tableofcontents

\chapter{Recenzent \odgovor{--- ocena: 5} }
	
	
	\section{O čemu rad govori?}
	Rad prezentuje GraalVM koji predstavlja virtuelnu mašinu razvijenu za potrebe efikasnog pisanja i izvršavanja kodova u različitim jezicima. Fokus je na komponentama ovog projekta i njihovom pojedinačnom značaju. Ističe se da je GraalVM nov kompajler koji ima puno potencijala da nadmaši sve svoje prethodnike. Navode se prednosti GraalVM-a kroz upotrebu biblioteke Truffle, kroz visoke performanse i dr. 
	% Напишете један кратак пасус у којим ћете својим речима препричати суштину рада (и тиме показати да сте рад пажљиво прочитали и разумели). Обим од 200 до 400 карактера.
	
	\section{Krupne primedbe i sugestije}
	Potpoglavlje 1.1 vezano za istorijat bi trebalo da bude odvojena celina mimo samog uvoda, dok potpoglavlje 1.2 ne mora postojati samo kao potpoglavlje, nego se treba uvrstiti u okviru poglavlja 1. 
	\odgovor{Smatramo da istorija zapravo objašnjava okolnosti u kojima je nastao projekat, što ga čini dobrim delom uvoda. Deo 1.2 je spojen sa Uvodom.}
    Potpoglavlje 2.4 je vidno obimnije od ostalih i potrebno ga je restruktuirati i potencijalno podeliti na više odvojenih podpoglavlja, dok je potpoglavlje 3.4 isuviše šturo, potrebno ga je potkrepiti sa još nekom rečenicom i dubljim objašnjenjem ili pak, u potuponosti izostaviti potpoglavlje i njegov sadašnji sadržaj spojiti sa već postojećim poglavljima na adekvatan način. \odgovor{Potpoglavlje 2.4. nije moguće podeliti na više potpoglavlja jer se celo odnosi na jednu zaokruženu celinu (\emph{Espresso}). Odlučili smo se i da ne dodajemo podpodnaslove unutar pomenutog potpoglavlja da ne bismo narušili simetričnost sadržaja rada. Poglavlje o naprednim alatima je izbačeno, dok se informacije i tabela sada nalaze u poglavlju Karakteristike.}
	% Напишете своја запажања и конструктивне идеје шта у раду недостаје и шта би требало да се промени-измени-дода-одузме да би рад био квалитетнији.
	
	\section{Sitne primedbe}
	% Напишете своја запажања на тему штампарских-стилских-језичких грешки
	\begin{enumerate}
		\item Poglavlje 1.1, poslednji pasus: umesto ''objavljen'' treba ''objavljena''.\\
		\odgovor{Popravljeno.}
		\item Poglavlje 1.2, prvi pasus: umesto ''kodom jednog jezika'' preciznije je reći ''kodom pisanim na jednom jeziku''.\\
		\odgovor{Popravljeno.}
		\item Poglavlje 1.2, šesti pasus: slovna greška u reči ''licensira''.\\
		\odgovor{Popravljeno.}
		\item Poglavlje 2.1, drugi pasus, druga rečenica: fali ''se'' kod ''da''.\\
		\odgovor{Popravljeno.}
		\item Poglavlje 2.1, treći pasus, prva rečenica: fali zarez nakon ''programi''.\\
		\odgovor{Popravljeno.}
		\item Poglavlje 2.2, četvrti pasus, druga rečenica: umesto ''sebe'' treba ''sebi''.\\
		\odgovor{Popravljeno.}
		\item Poglavlje 2.2, četvrti pasus, četvrta rečenica: umesto ''Ovim omogućavamo''	treba ''Ovim se omogućava''.\\
		\odgovor{Popravljeno.}
		\item Poglavlje 2.4, treći pasus: izraz ''Svetim gralom''  staviti pod navodnike.\\
		\odgovor{Popravljeno.}
		\item Poglavlje 2.4, četvrti pasus: fali slovo u reči ''upostavi''.\\
		\odgovor{Popravljeno.}
		\item Poglavlje 2.4: pominje se termin ''poliglot programiranje'' koje je u radu tek kasnije objašnjeno.\\
		\odgovor{Dodata je fusnota koja objašnjava termin i referiše poglavlje u kom će biti više reči o tome.}
		\item Poglavlje 4, podnaslov Nvidia: umesto ''GraalVM'' staviti adekvatni nastavak ''GraalVM-u/GraalVM-a''.\\
		\odgovor{Popravljeno.}
		\item Zaključak, poslednji pasus: umesto ''pojedinačna'' treba ''pojedinačne'', fali slovo u reči ''konkurncija''.\\
		\odgovor{Popravljeno.}
	\end{enumerate}
	
	\section{Provera sadržajnosti i forme seminarskog rada}
	% Oдговорите на следећа питања --- уз сваки одговор дати и образложење
	
	\begin{enumerate}
		\item Da li rad dobro odgovara na zadatu temu?\\
		Da, rad se fokusira na koncizna objašnjenja pojedinačnih komponenti i brojnih prednosti ovog projekta, kao i na trenutni nivo napretka i upotrebe.
		\item Da li je nešto važno propušteno?\\
		Ne, sve teme iz opisa su pokrivene, kao i neke druge.
		\item Da li ima suštinskih grešaka i propusta?\\
		Ne, rad pokriva sva istaknuta pravila i opravdava očekivanja.
		\item Da li je naslov rada dobro izabran?\\
		Da. Tema je konkretna i drugačiji naslov bi bio suvišan.
		\item Da li sažetak sadrži prave podatke o radu?\\
		Da, na jednostavan i precizan način uvodi i budi interesovanje za dalje čitanje.
		\item Da li je rad lak-težak za čitanje?\\
		Rad je lak za čitanje, pregledan i pisan tako da jedna tema prati drugu, čineći sve jednom celinom. 
		\item Da li je za razumevanje teksta potrebno predznanje i u kolikoj meri?\\
		Da, definitivno je potrebno predznanje, naročito za nekog ko je laik; tema je sama po sebi jako ustručena.
		\item Da li je u radu navedena odgovarajuća literatura?\\
		Da, literatura je adekvatna; korišćene su validne veb stranice, naučni radovi su sa stručnih sajtova, a i knjiga je uvrštena u literaturu.
		\item Da li su u radu reference korektno navedene?\\
		Da, reference su stavljane na prava mesta u okviru rečenica i postavljene su u dovoljnoj meri; veći deo teksta je njima potkrepljen. 
		\item Da li je struktura rada adekvatna?\\
		Većinom da, međutim propusti su navedeni u poglavlju o krupnim primedbama i sugestijama, jer bi ovim izmenama rad bio strukturno puno bolji. \odgovor{Odgovor na ovo se nalazi u odgovoru na Krupne primedbe.}
		\item Da li rad sadrži sve elemente propisane uslovom seminarskog rada (slike, tabele, broj strana...)?\\
		Da, svi elementi su prisutni i u adekvatnom formatu.
		\item Da li su slike i tabele funkcionalne i adekvatne?\\
		Da, naročito ističem slike koje daju dinamiku i živopisnost radu, a uz to pružaju adekvatne informacije.
	\end{enumerate}
	
	\section{Ocenite sebe}
	% Napišite koliko ste upućeni u oblast koju recenzirate: 
	% a) ekspert u datoj oblasti
	% b) veoma upućeni u oblast
	% c) srednje upućeni
	% d) malo upućeni 
	% e) skoro neupućeni
	% f) potpuno neupućeni
	% Obrazložite svoju odluku
	U ovu oblast sam srednje upućena; rad sam apsolutno razumela, jer je većina tema pokrivena znanjem koje stičemo na fakultetu, čak i ono sto nije pokriveno fakultetskim obrazovanjem je lako savladivo, upravo zbog predznanja koje nam omogućava lako prilagođavanje novinama u modernom svetu programiranja.
	
	
\chapter{Recenzent \odgovor{--- ocena: 5} }


\section{O čemu rad govori?}
% Напишете један кратак пасус у којим ћете својим речима препричати суштину рада (и тиме показати да сте рад пажљиво прочитали и разумели). Обим од 200 до 400 карактера.
Rad opisuje projekat GraalVM koji predstavlja JDK visokih performansi namenjen da se ubrzaju programi pisani u nekom od jezika Java porodice.
Opisane su glavne prednosti projekta u odnosu na tradicionalni JVM pristup i to: korišćenje više programskih jezika u okviru iste aplikacije,
kao i različiti pristupi prevođenja Java-e(AOT i JIT) kao i osnovne primene projekta pre svega u cloud industriji.

\section{Krupne primedbe i sugestije}
% Напишете своја запажања и конструктивне идеје шта у раду недостаје и шта би требало да се промени-измени-дода-одузме да би рад био квалитетнији.
Jedna od ključnih zamerki je uvodna rečenica koja glasi: \textit{GraalVM je alat koji omogućava pisanje i izvršavanje koda u različitim
jezicima}. Na zvaničnoj stranici projekta, GraalVM je opisan kao JDK visokih performansi namenjen da se ubrzaju programi napisani u Javi.
Takođe, projekat nudi mogućnost prevođenja Java programa na dva načina:
\begin{itemize}
    \item GraalVM JIT kompajlera
    \item AOT tehnologije, kojom se dobija binarna izvršna datoteka (Native Image)
\end{itemize}
Kao jedna od komponenti izdvaja se i Truffle radni okvir(eng.~{\em Framework}) koji omogućava korisnicima da pišu višejezične aplikacije 
(eng.~{\em polyglot applications}). 
Dakle, GraalVM ne predstavlja alat za pisanje i izvršavanje koda u različitim jezicima, već JDK koji nudi gore pomenute karakteristike(što 
predstavlja daleko više od alata za pisanje višejezičnih aplikacija).

\odgovor{
Uvod je izmenjen na sledeći način:
GraalVM predstavlja JDK (\emph{Java Development Kit}) visokih performansi \cite{graalvmintroduction}. GraalVM omogućava ubrzanje izvršavanja uz korišćenje manje resursa. Nudi prevođenje Java aplikacija na dva načina: AOT (\emph{Ahead Of Time}) i JIT (\emph{Just In Time}).
}

U sekciji 2, navedeno je sledeće: \textit{Detaljnija testiranja su ipak pokazala da se u proseku Ruby kod izvšava oko 30\% brže na GraalVM}. 
Kao potkrepljenje ove konstatacije naveden je link ka sledećoj referenci: \href{https://www.graalvm.org/}{https://www.graalvm.org/}. Na 
navedenoj referenci nije moguće(barem ne jednostavno) pronaći ovaj podatak. Sugestija bi bila da referenca bude nešto konkretnija, kako bi se 
potvrdila tačnost navedene tvrdnje. 
\odgovor{Ispravljena je referenca.}

Rad je mogao biti nešto drugačije struktuiran, gde bi se u sekciji 2: (\textit{Šta obuhvata projekat?}) svakako trebali navesti GraalVM JIT 
kompajler i Native image tehnologija(navedena u sekciji 3, koja opisuje karakteristike projekta). 
\odgovor{Sekcija je prebačena. Sve što se odnosi na Native Image sada je struktuirano na bolji način i nalazi se u potpoglavlju 2.3. U pomenutom poglavlju dodat je još jedan (treći) pasus.}

Takođe, možda je trebalo prebaciti težište rada sa mogućnost pisanja višejezičnih aplikacija, na mesta na kojima se projekat može primeniti(i 
zašto se na specifičnim mestima primene GraalVM-a ostvaruju bolji rezultati nego kada se koristi JVM). Svakako jedna od ključnih prednosti 
GraalVM-a predstavlja unapređenje performansi Java jezika u određenim situacijama (najočigledniji primer je korišćenje GraalVM-a za cloud 
aplikacije zbog niskog vremena zagrevanja (eng.~{\em warm-up time})). Ova stavka pomenuta je, ali procentualno daleko manje u odnosu na 
višejezičnost.
\odgovor{Smatramo da ovaj rad navodi osnovne elemente, karakteristike i ciljeve GraalVM-a. Višejezničnost je nešto specifično za GraalVM zbog čega je u radu tome posvećeno više pažnje. Smatramo da u radu postoji osvrt na performanse i primene GraalVM-a. U delu \emph{3.1 Visoke performanse} je benčmark testiranjem pokazano koliko je GraalVM performantniji u odnosu na neke druge JVM-ove.}

\section{Sitne primedbe}
% Напишете своја запажања на тему штампарских-стилских-језичких грешки
U nekim situacijama, suština je ispravno napisana, ali način na koji su rečenice organizovane može zvučati komplikovano. Tako na primer: 
\textit{GraalVM pruža tehnologiju pod nazivom Native Image [8]. Native Image predstavlja izvršivi binarni fajl koji je dobijen GraalVM AOT 
kompilacijom}. U prvoj rečenici, termin Native Image upotrebljen je kao tehnologija, a već u narednoj rečenici Native Image je izvršni fajl 
dobijen AOT kompilacijom. Suština je ispravna ali bi se rečenica mogla formulisati nešto drugačije, recimo \textit{"GraalVM pruža tehnologiju 
Native Image. Ova tehnologija u pozadini koristi AOT kompajler kako bi se od Java programa dobila izvršiva biblioteka, takođe nazvana native 
image."} Ako bi se pomnije obratila pažnja, u zvaničnoj dokumentaciji, kada se govori o Native Image-u kao tehnologiji, naziv se navodi sa oba 
velika slova. S druge strane, kada se govori o izvršivoj datoteci, ime se navodi malim početnim slovima. 
Potkrepljenje činjenice: \href{https://www.graalvm.org/22.0/reference-manual/native-image/}{https://www.graalvm.org/22.0/reference-manual/
native-image/} \odgovor{Promakla nam je ta informacije, sada je rečenica ispravljena. Takođe, ispravljena su i sva ostala mesta gde se nalazi \emph{native image} koji se odnosi na izvršivu datateku.}

\indent Reference koje ukazuju na veb stranice, bi trebale da budu nešto konkretnije. Većina navedenih referenci ukazuje samo na početnu stranicu, na kojoj se u većini slučajeva ne nalaze podaci koji su referencirani.  \odgovor{Dodate su reference koje vode na određene stranice sajta GraalVM.org, to su stranice Introduction, Native Image, Java on Truffle.}

\indent Rad sadrži dosta termina koji su prevedeni bez adekvatnog stranog termina. \odgovor{Popravljeno. Na svim mestima gde je prepoznato da je potrebno dodat je odgovarajući engleski termin.}

\indent Na par mesta postoje sitne slovne greške(npr. JKD umesto JDK). \odgovor{Popravljeno.}

\section{Provera sadržajnosti i forme seminarskog rada}
% Oдговорите на следећа питања --- уз сваки одговор дати и образложење

\begin{enumerate}
\item Da li rad dobro odgovara na zadatu temu?\\
U načelu rad sadrži sve ključne delove koje projekat sadrži, uz primedbu da se akcenat stavio na neke manje bitne delove projekta u odnosu na 
neke ključne delove projekta. Akcenat rada je na višejezičnosti a ne na specifičnom pristupu prevođenja Java jezika i situacijama u kojima bi 
se GraalVM pre koristio nego tradicionalni JVM. 
\odgovor{Odgovor na ovo se nalazi u odgovoru na Krupne primedbe.}

\item Da li je nešto važno propušteno?\\
Sve teme su pokrivene, uz primedbu da je o Native Image tehnologiji posvećen samo jedan pasus, a upravo ova tehnologija predstavlja jako bitan 
deo projekta GraalVM. \odgovor{Popravljeno. Native Image-u sada je posvećeno potpoglavlje 2.3.}

\item Da li ima suštinskih grešaka i propusta?\\
Osim uvodne rečenice, u kojoj je sam GraalVM pogrešno definisan, ostatak projekta nema suštinskih grešaka. \odgovor{Popravljeno. Uvodna rečenica je izmenjena na način kako je navedeno u odgovoru na Krupne primedbe.}

\item Da li je naslov rada dobro izabran?\\
Navedeni naslov odgovara temi koja je opisana u projektu.

\item Da li sažetak sadrži prave podatke o radu?\\
Kao i uvodna rečenica, sažetak stavlja akcenat na višejezičnost, te je poželjno modifikovati ga, tako da se akcenat pomeri bliže definiciji da 
GraalVM predstavlja JDK visokih performansi. \odgovor{Odgovor na ovo se nalazi u odgovoru na Krupne primedbe. U sažetku je pored višejezičnosti pomenuta i efikasnost i široka primena GraalVM-a.}

\item Da li je rad lak-težak za čitanje?\\
Obzirom na kompleksnost projekta o kojem rad govori, rad sadrži dosta stručne terminologije, što ga čini težim za čitanje(što nije primedba 
autorima rada, već posledica kompleksnosti projekta).

\item Da li je za razumevanje teksta potrebno predznanje i u kolikoj meri?\\
Kao što je pomenuto u prethodnoj tačci, rad sadrži dosta stručne terminologije, i opisa tehnologija koje se ne koriste u velikoj meri(ili barem 
u obliku u kojem se koriste u ovom projektu), te je za njegovo razumevanje potrebno značajno predznanje.

\item Da li je u radu navedena odgovarajuća literatura?\\
Literatura je odgovarajuća uz već navedenu primedbu da bi reference ka veb lokacijama mogle biti konkretnije(ka konkretnoj a ne ka početnoj 
strani). \odgovor{Popravljeno, Odgovor se nalazi u odgovoru na Sitne primedbe.}

\item Da li su u radu reference korektno navedene?\\
Pomenuto u prethodnoj tačci.

\item Da li je struktura rada adekvatna?\\
Određene delove bi trebalo zameniti, kako je to već navedeno. Primer: Native Image prebaciti u sekciju 2(Šta projekat obuhvata), dok bi se 
podaci o performansama(konkretno za Ruby jezik) mogle izmestiti u sekciju 3(Karakteristike). \odgovor{Native Image je prebačen u sekciju 2 (kao potpoglavlje 2.3). Odlučili smo da u sekciji budu opisani konkretni rezultati benčmark testiranja. S obzirom da su performanse za Ruby jezik samo pomenute ostavljene su potpoglavlju 2.1.}

\item Da li rad sadrži sve elemente propisane uslovom seminarskog rada (slike, tabele, broj strana...)?\\
Da. Rad sadrži sve tražene elemente i adekvatan broj strana. 

\item Da li su slike i tabele funkcionalne i adekvatne?\\


\end{enumerate}

\section{Ocenite sebe}
% Napišite koliko ste upućeni u oblast koju recenzirate: 
% a) ekspert u datoj oblasti
% b) veoma upućeni u oblast
% c) srednje upućeni
% d) malo upućeni 
% e) skoro neupućeni
% f) potpuno neupućeni
% Obrazložite svoju odluku
Smatram da sam srednje upućen u oblast koju recenziram jer sam jako zainteresovan za projekat o kojem rad govori.



\chapter{Dodatne izmene}
%Ovde navedite ukoliko ima izmena koje ste uradili a koje vam recenzenti nisu tražili. 
[Formatiranje; Uvod i Zaključajk]: Neki paragrafi su spojeni u jedan, što čini rad preglednijim.

\end{document}
