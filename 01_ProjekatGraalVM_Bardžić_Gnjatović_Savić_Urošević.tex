% !TEX encoding = UTF-8 Unicode
\documentclass[a4paper]{article}

\usepackage{color}
\usepackage{url}
\usepackage[T2A]{fontenc} % enable Cyrillic fonts
\usepackage[utf8]{inputenc} % make weird characters work
\usepackage{graphicx}

\usepackage[english,serbian]{babel}
%\usepackage[english,serbianc]{babel} %ukljuciti babel sa ovim opcijama, umesto gornjim, ukoliko se koristi cirilica

\usepackage[unicode]{hyperref}
\hypersetup{colorlinks,citecolor=green,filecolor=green,linkcolor=blue,urlcolor=blue}

\usepackage{listings}

%\newtheorem{primer}{Пример}[section] %ćirilični primer
\newtheorem{primer}{Primer}[section]

\definecolor{mygreen}{rgb}{0,0.6,0}
\definecolor{mygray}{rgb}{0.5,0.5,0.5}
\definecolor{mymauve}{rgb}{0.58,0,0.82}

\lstset{ 
  backgroundcolor=\color{white},   % choose the background color; you must add \usepackage{color} or \usepackage{xcolor}; should come as last argument
  basicstyle=\scriptsize\ttfamily,        % the size of the fonts that are used for the code
  breakatwhitespace=false,         % sets if automatic breaks should only happen at whitespace
  breaklines=true,                 % sets automatic line breaking
  captionpos=b,                    % sets the caption-position to bottom
  commentstyle=\color{mygreen},    % comment style
  deletekeywords={...},            % if you want to delete keywords from the given language
  escapeinside={\%*}{*)},          % if you want to add LaTeX within your code
  extendedchars=true,              % lets you use non-ASCII characters; for 8-bits encodings only, does not work with UTF-8
  firstnumber=1000,                % start line enumeration with line 1000
  frame=single,	                   % adds a frame around the code
  keepspaces=true,                 % keeps spaces in text, useful for keeping indentation of code (possibly needs columns=flexible)
  keywordstyle=\color{blue},       % keyword style
  language=Python,                 % the language of the code
  morekeywords={*,...},            % if you want to add more keywords to the set
  numbers=left,                    % where to put the line-numbers; possible values are (none, left, right)
  numbersep=5pt,                   % how far the line-numbers are from the code
  numberstyle=\tiny\color{mygray}, % the style that is used for the line-numbers
  rulecolor=\color{black},         % if not set, the frame-color may be changed on line-breaks within not-black text (e.g. comments (green here))
  showspaces=false,                % show spaces everywhere adding particular underscores; it overrides 'showstringspaces'
  showstringspaces=false,          % underline spaces within strings only
  showtabs=false,                  % show tabs within strings adding particular underscores
  stepnumber=2,                    % the step between two line-numbers. If it's 1, each line will be numbered
  stringstyle=\color{mymauve},     % string literal style
  tabsize=2,	                   % sets default tabsize to 2 spaces
  title=\lstname                   % show the filename of files included with \lstinputlisting; also try caption instead of title
}

\begin{document}

\title{Projekat GraalVM\\ \small{Seminarski rad u okviru kursa\\Metodologija stručnog i naučnog rada\\ Matematički fakultet}}

\author{Bojan Bardžić, Milica Gnjatović, Pavle Savić, Andrija Urošević\\ kontakt email prvog, drugog, trećeg, \texttt{mi18083@alas.matf.bg.ac.rs}}

%\date{9.~april 2015.}

\maketitle

\abstract{
U ovom tekstu je ukratko prikazana osnovna forma seminarskog rada. Obratite pažnju da je pored ove .pdf datoteke, u prilogu i odgovarajuća .tex datoteka, kao i .bib datoteka korišćena za generisanje literature. Na prvoj strani seminarskog rada su naslov, apstrakt i sadržaj, i to sve mora da stane na prvu stranu! Kako bi Vaš seminarski zadovoljio standarde i očekivanja, koristite uputstva i materijale sa predavanja na temu pisanja seminarskih radova. Ovo je samo šablon koji se odnosi na fizički izgled seminarskog rada (šablon koji \emph{morate} da koristite!) kao i par tehničkih pomoćnih uputstava. Pročitajte tekst pažljivo jer on sadrži i važne informacije vezane za zahteve obima i karakteristika seminarskog rada.}

\tableofcontents

\newpage

\section{Uvod}
\label{sec:uvod}
GraalVM je alat koji omogućava pisanje i izvršavanje java koda. Pored jave GraalVM podržava i druge jezike: 
\begin{itemize}
	\item JavaScript i Node.js
	\item Python
	\item Ruby
	\item R
	\item LLVM jezici poput C-a i C++-a
	\item WebAssembly
\end{itemize}

GraalVM je implementiran u javi. \\

GraalVM je koristan i u radu sa mikroservisnom aritekturom. Nekolicina okvira za rad sa Java mikroservisima  (Java microservice frameworks) je već prihvatilo ovu platformu. Između ostalih to su Micronaut, Spring, Helidon i Quarkus. \\

Još jedna od mogućnosti koje GraalVM nudi je implemenitranje novih jezika i alata korišćenjem biblioteke Truffle.

Korišćenjem ovog alata je omoućeno efikasnije korišćenje više jezika na jednom projektu. Pri korišćenju više jezika na projektu je moguće kodom jednom jezika pozivati funkcije koje su pisane u drugom jeziku. Dopušteno je deljenje struktura podataka između kodova pisanih u različitim jezicima. Na ovaj način je omogućeno da sakupljač otpadaka radi na celom projektu iako se satoji iz više jezika, kao i jednostavnije debagovanje.
GraalVM čine Java Virtuelna Mašina - JVM i Java Development Kit - JDK. Ovaj alat je proizvod kompanije Oracle. Njegov cilj je da se omogući brže izvršavanje koda, a da se pri tome koristi manje memorije.

Ovaj alat je dostupan na Linux, Windows i macOS operativnim sistemima. \\

GraalVM je dostupan kao Community i Enterprise edition. Community edition je otvorenog koda i dostupna je na gituhb repozitorijumu (https://github.com/oracle/graal). Enterprise Edition razvija i licensira kompanije Oracle. \\
Dva osnovna načina na koje se može doprineti projektu GraalVM je prijavljiavanjem github issue-a i pravljenjem pull requestova. \\

GraalVM podržavaju razvojna okruženja i protokoli za debagovanje. Neka od podržanih okruženja su Eclipse, NetBeans, IntellliJ IDEA, Visual Studio Code. Ova okruženja su posebno dobra je podržaju sve jezike koje podržava i GraalVM. GraalVM obezbeđuje ugrađen Crome DevTools Protocol, Debug Adapter Protocol (DAP) i Language Server Protocol (LSP), čime je omogućeno debagovanje JavaScript, R i Ruby kodova.

\subsection{Istorijski razvoj}
\label{subsec:Istorijski razvoj}

Java Virtuelne mašine poput Oraklovog Java HotSpotVM i IBMov Java VM postoje već 15ak godina. Međutim za druge jezike nije postojalo tako nešto. Cilj ovog projekta je bio da se napavi objedinjena virtuelna mašina koja bi mogla efikasno da izvršava kodove različtih jezika.\\

GraalVM 19.0 je objavljen u Maju 2019e i to je prva verzija ovog alata. Trenutno najnovija verzija je GraaLVM 22.1.0 objavljena u Aprilu 2022. 

\section{Karakteristike}
\label{sec:Karakteristike}

\subsection{Native Image}
\label{sub:Native Image}

\subsection{Napredni optimizator}
\label{sub:Napredni optimizator}

\subsection{Polygot}
\label{sub:Polygot}


\subsection{Napredni alati}
\label{sub:Napredni alati}


\section{Zaključak}
\label{sec:zakljucak}

\addcontentsline{toc}{section}{Literatura}
\appendix
\bibliography{seminarski} 
\bibliographystyle{plain}

\appendix
\section{Dodatak}
Ovde pišem dodatne stvari, ukoliko za time ima potrebe.
Ovde pišem dodatne stvari, ukoliko za time ima potrebe.
Ovde pišem dodatne stvari, ukoliko za time ima potrebe.
Ovde pišem dodatne stvari, ukoliko za time ima potrebe.
Ovde pišem dodatne stvari, ukoliko za time ima potrebe.


\end{document}
