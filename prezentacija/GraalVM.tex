\documentclass{beamer}
\usetheme{Madrid}
%\usecolortheme{beaver}
\usepackage[utf8x]{inputenc}
\usepackage[serbian]{babel}
\usepackage{verbatim}
\usepackage{amsmath}
\usepackage{graphicx}
\usepackage{caption}


\title{GraalVM}
\subtitle{predmet: Metodologija stručnog i u naučnog rada}
\author{Bojan Bardžić \and Milica Gnjatović \and Pavle Savić \and Andrija Urošević}
\institute{Matematički fakultet}
\date{}

\usepackage{fancyhdr}

\begin{document}
	\begin{frame}
		\titlepage
	\end{frame}
	
	\section{Uvod}
	\begin{frame}
		\frametitle{Uvod}
		
		\begin{figure}
			\begin{center}
				\includegraphics[width=0.5\linewidth]{imgs/graalvm_logo.png}	
			\end{center} 
		\end{figure}
	
		\center	
		\textit{high-performance JDK distribution written for Java and other JVM languages}

		\begin{flushleft}
			\begin{itemize}
				\item \emph{One VM to rull them all}
				\item Oracle Labs
				\item Community i Enterprise Edition
				\item JVM + JDK	
				\item JIT i AOT
			\end{itemize}
		\end{flushleft}


	\end{frame}

	\begin{frame}
		\frametitle{Podržani jezici}
		\begin{figure}
			\includegraphics[width=0.16\linewidth]{imgs/java_logo.png}
			\includegraphics[width=0.16\linewidth]{imgs/c_logo.png}
			\includegraphics[width=0.16\linewidth]{imgs/cpp_logo.png}
			\includegraphics[width=0.16\linewidth]{imgs/r_logo.png}
			\includegraphics[width=0.16\linewidth]{imgs/python_logo.png}
			\includegraphics[width=0.16\linewidth]{imgs/ruby_logo.png}
			\includegraphics[width=0.16\linewidth]{imgs/js_logo.png}
			\includegraphics[width=0.16\linewidth]{imgs/nodejs_logo.png}
			\includegraphics[width=0.16\linewidth]{imgs/clojure_logo.png}
			\includegraphics[width=0.16\linewidth]{imgs/kotlin_logo.png}
			\includegraphics[width=0.16\linewidth]{imgs/scala_logo.png}
			\includegraphics[width=0.16\linewidth]{imgs/wa_logo.png}
		\end{figure}
		
		\begin{flushleft}
			\begin{itemize}
				\item Eclipse, NetBeans, IntelliJ IDEA i Visual Studio Code
				\item Windows, Linux, MacOS
			\end{itemize}
		\end{flushleft}
		
	
	\end{frame}	
	

	\section{Šta obuhvata projekat?}
	
	\begin{frame}
		\frametitle{Zamena za JVM}
		\begin{itemize}
			\item Može se koristiti umesto JVM
			\item Za Ruby do 30\% brži
			\item Za Javu sličan HotSpot VM
			\item Pisan u Javi
			\item Veliki potencijal za optimizaciju
		\end{itemize}
	\end{frame}	
	
	\begin{frame}
		\frametitle{Native Image}
		\begin{itemize}
			\item AOT kompajler za Javu
			\item Mikroservisi na cloud-u
			\item Manje zauzeće memorije
			\item Brže izvršavanje
			\item Ne može se primeniti na sve Java programe
		\end{itemize}
	\end{frame}	
	
	
	\begin{frame}
		\frametitle{Truffle}
		 \begin{itemize}
			\item Omogućava pravljenje alata i interpretera
			\item Parcijalna evaluacija
			\item Podrška za Ruby, R, Python
			\item Podrška za LLVM jezike
			\item Omogućava poliglotsko programiranje
		 \end{itemize}
	\end{frame}	

	\begin{frame}
		\frametitle{Espresso - Java on Truffle}

		\begin{figure}
			\begin{center}
				\includegraphics[width=0.5\linewidth]{imgs/java_on_truffle.png}	
			\end{center} 
		\end{figure}

		\center	
		\textit{Java on Truffle u GraalVM ekosistemu}


		\begin{flushleft}
			\begin{itemize}
				\item Dostupna od verzije 21.0
				\item Implementacija JVM specifikacije pomoću Truffle-a
				\item Interoperabilnost sa interpretiranim i LLVM jezicima
				\item \emph{Self-hosting}, metacirkularna VM
				\item Izvorni kod razumljiv Java programerima
				\item Istovremeno JVM i Java program
			\end{itemize}
		\end{flushleft}

	\end{frame}	

	\begin{frame}
		\frametitle{Espresso - Java on Truffle}

		\begin{flushleft}
			\begin{itemize}
				\item Moguće ugraditi Java 8 kontekst u Java 11 aplikaciju
				\item Povećana izolovanost \emph{host} VM-a i Java programa
				\item Napredni \emph{hot swap} prilikom debagovanja
				\item Eksperimentalna tehnologija
			\end{itemize}
		\end{flushleft}


		\begin{figure}
			\begin{center}
				\includegraphics[width=0.5\linewidth]{imgs/hotswap.png}	
			\end{center} 
		\end{figure}

		\center	
		\textit{Hot Swap u IntelliJ IDEA debageru}

	\end{frame}
	
		
	\section{Karakteristike}
	
	\begin{frame}
		\frametitle{Visoke performanse}
	\end{frame}	

	\begin{frame}
		\frametitle{Poliglot programiranje}
	\end{frame}	

	\begin{frame}
		\frametitle{Napredni alati}
	\end{frame}	

	\section{Zaključak}
	
	\begin{frame}
		\frametitle{Ko koristi GraalVM?}
		\begin{figure}
			\includegraphics[width=0.2\linewidth]{imgs/facebook.png}
			\includegraphics[width=0.25\linewidth]{imgs/twitter.png} 
			\includegraphics[width=0.25\linewidth]{imgs/nvidia.png} \\
			\includegraphics[width=0.25\linewidth]{imgs/gs.png}
			\includegraphics[width=0.25\linewidth]{imgs/politie.png}
			\includegraphics[width=0.25\linewidth]{imgs/oracle.png}
		\end{figure}
	\end{frame}
	
	\begin{frame}
		\center	
		\textit{Hvala na pažnji!}
	\end{frame}

	\begin{frame}
		\nocite{*}
		\bibliography{GraalVM} 
		\bibliographystyle{abbrv}
	\end{frame}
	
\end{document}